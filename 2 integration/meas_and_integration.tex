\section{利用积分理论解决测度问题}
``实分析"是个庞大的体系, 分析学中的各种理论与技巧都可以塞进这个框.
目前大家可能觉得测度就是测度, 积分就是积分, 但它们可以由特征函数自然地联系起来:
$$m(E) = \int_{\R^d} \chi_{E}(x) dx.$$
学到抽象测度时, 我们会知道: 积分只是测度的一个例子. 

许多问题的题干中只有勒贝格测度, 看起来像是只用测度的一堆性质就能做出来. 实际上, 这些题需要利用特征函数联系测度与积分, 再利用积分的性质解决. 
\begin{example} %UW-Madison Qual
    设$E \subset \R$可测且具有有限测度. 定义函数$f:\R \to \R$,
    $f(x)=m(E \cap(E+x))$. 证明$f$连续.
\end{example}
\begin{proof}
    
\end{proof}

\begin{example} %UW-Madison Qual
    Let $E \subset[0,1]$ be a Lebesgue measurable set with $m(E)>0.999$. Prove that there exists $\epsilon_0>0$ such that for every $t \in\left(0, \epsilon_0\right)$, we can always find $x$ such that $x, x+t, x+t^2 \in E$.

    Hint: Try estimating $\int_0^1\left(\chi_E(x)+\chi_E(x+t)+\chi_E\left(x+t^2\right)\right) d x$, where $\chi_E$ is the characteristic function of $E$.
\end{example}

\begin{example} %UW-Madison Qual
    Let $E \subset[0,1]$ be a measurable set with positive Lebesgue measure. Let $\chi$ be the characteristic function of $E$.
    (a) Let $F(x)=\int_{\mathbb{R}} \chi(x-t) \chi(t) d t$. Prove that $F$ is a continuous function.
    (b) Let $E+E=\left\{e_1+e_2 \in \mathbb{R} \mid e_1 \in E, e_2 \in E\right\}$. Prove that $E+E$ contains a non-empty open subset of $\mathbb{R}$.
\end{example}
