\section{可测函数}
\subsection{可测函数类的封闭性}
设$\{f_n: n \in \N\}$是一列可测函数, 则下列函数均可测:
\begin{enumerate}
{\everymath{\displaystyle}
    \item $\sup_{n \in \N}f_n$;
    \item $\inf_{n \in \N} f_n$;
    \item $\limsup_{n \in \N} f_n$;
    \item $\liminf_{n \in \N} f_n$.}
\end{enumerate}
\begin{remark}
    输入一个$x$, $\sup_n f_n(x)$就是一个数列的上确界. 
    这样$\sup_n f_n$就确定了一个函数: 
    $$\left(\sup_n f_n\right)(x) = \sup_n \{f_n(x)\}.$$
\end{remark}
若$f_n \to f$ a.e., 则$\limsup f_n = \liminf f_n = f$ a.e., 从而$f$可测. 我们得出可测函数类对(逐点)极限运算封闭, 连续函数做得到吗? 

\begin{exercise}
    构造一个可测函数$f$和一个连续函数$\phi$, 使得$f \circ \phi$不可测.
    以此构造证明存在不是博雷尔集的勒贝格可测集, 因而勒贝格$\sigma$-代数真包含博雷尔$\sigma$-代数
\end{exercise}

\subsection{验证可测函数}
\begin{example}[~(Dini导数)] % Stein ex 3.14
    设$F: [a,b] \to \R$连续. 证明
    $$D^+(F)(x) = \limsup_{h \to 0^+}\frac{F(x+h)-F(x)}{h}$$
    可测. 
\end{example}
\begin{solution}
    我们要把连续型的极限转化为离散型的极限, 这样才能利用可测函数类的封闭性. 根据函数上极限
    \footnote{为了不让读者翻到前面寻找, 打断阅读, 故将定义再复习一遍: $\limsup_{h \to 0}f(h) = \lim_{\delta \to  0} \sup_{0<|h|<\delta} f(h)$.}
    的定义,
    $$\limsup_{h \to 0^+}\frac{F(x+h)-F(x)}{h} = 
    \lim_{\delta \to 0}\sup_{0<h<\delta}\frac{F(x+h)-F(x)}{h}.$$
    将极限右侧的上确界看作一个整体
    $\displaystyle{G(\delta)=\sup_{0<h<\delta}\frac{F(x+h)-F(x)}{h} }$, 则$G$单调增, 所以其极限存在(包括$\infty$). 这样我们就可以取$\delta=1/n \to 0$, 完成第一次离散化:
    $$\limsup_{h \to 0^+}\frac{F(x+h)-F(x)}{h} = 
    \lim_{n \to \infty}\sup_{0<h<1/n}\frac{F(x+h)-F(x)}{h}.$$
    现在只需证明对每个$n \in \N$, $\displaystyle{\sup_{0<h<1/n}\frac{F(x+h)-F(x)}{h}}$可测.
    但是, 该上确界是对不可数个$h$取的, 所以我们要把“卡池”转化为可数的. 回忆到连续函数的值完全由其在有理数上的值决定, 所以猜测:
    $$A := \sup_{0<h<1/n}\frac{F(x+h)-F(x)}{h}=\sup_{h \in (0, 1/n) \cap \Q}
    \frac{F(x+h)-F(x)}{h} := B.$$
    由于$(0,1/n)\cap \Q \subset (0, 1/n)$, 得$A \geq B$. 
    记$\displaystyle{f(h)=\frac{F(x+h)-F(x)}{h}}$, 由$A=\sup_{0<h<1/n}f(h)$可知存在数列$\{x_j\} \subset (0,1/n)$使$f(x_j) \to A$. 如果我们能找到$(0,1/n)$中的有理数列$\{y_j\}$使得$f(y_j) \to A$, 就可以证明命题. 根据有理数集的稠密性, 对每个$x_j$, 都有有理数列$\{y_{k,j}:k \in \N \} \subset (0, 1/n) \cap \Q$满足$y_{k,j} \to x_j (k \to \infty)$. 由$f$的连续性可知$f(y_{k,j}) \to f(x_j) (k \to \infty)$. 
    $$\begin{matrix}
        f(y_{1,1}) & f(y_{1,2}) & f(y_{1,3}) & \cdots & \\
        f(y_{2,1}) & f(y_{2,2}) & f(y_{2,3}) & \cdots & \\
        f(y_{3,1}) & f(y_{3,2}) & f(y_{3,3}) & \cdots & \\
        \vdots  & \vdots  & \vdots  &  \\
        \downarrow & \downarrow & \downarrow &  \\
        f(x_1)     & f(x_2)     & f(x_3)     & \cdots & \to A
    \end{matrix}$$
    面对这种二重极限, 我们应用Cantor对角化法则(Cantor diagonalization), 取$\{y_j\}=\{y_{j,j}\}$, 便得$f(y_{j}) \to A$, 从而$B\geq A$, 所以$A=B$. 
\end{solution}
\textbf{总结}: 
    若将$(0,1/n) \cap \Q$的点一一列出: $\{a_1, a_2, \cdots\}$, 我们可得可测函数列
    $f(a_n)=(F(x+a_n)-F(x))/a_n$, 可记作$g_n(x)=(F(x+a_n)-F(x))/a_n$. 一开始的上极限就可写成
    $$\lim_{n\to \infty}\sup_{n \in \N} g_n(x),$$
    这显然是一个可测函数. 
    
\begin{knowledge}
    上极限的定义; 函数极限的序列形式; 可测函数类得封闭性; Cantor对角化法则
\end{knowledge}

\begin{exercise} % Stein ex 1-21
    证明存在一个连续函数将一个勒贝格可测集映射至一个不可测集. (提示: 考虑$[0,1]$的一个不可测子集, 以及它在康托-勒贝格函数下的原像)
\end{exercise}

\subsection{Borel-Cantelli引理}

\begin{exercise}\footnote{Real Analysis, Stein, 1.17}
    设$\{f_n\}$为$[0,1]$上的一列可测函数, 并且$|f_n(x)|$几乎处处有限. 证明: 存在正实数列$\{c_n\}$满足
    $$ \frac{f_n(x)}{c_n} \to 0 \quad a.e. x. $$
\end{exercise}

\subsection{随机变量与分布函数}
可测函数真正有趣的地方在其于概率论中的伪装: 随机变量. 
设$(\Omg, \calF, \P)$为一概率空间, 若$X:\Omg \to \R$(博雷尔)可测, 即$X^{-1}(B) \in \calF, ~\forall B \in \calB_\R$, 那么称$X$为一个\textbf{随机变量}(random variable). 
\begin{example}
    要验证$X$是否可测, 其实不需要检查每个$B \in \calB_\R$, 只需检查$\calB_\R$的生成集即可. 证明: 若$X^{-1}(E) = \{\omg:X(\omg) \in E\} \in \calF$对所有$E \in \calE$成立, 且$\sigma(\calE) = \calB_\R$, 那么$X$可测. 
\end{example}
注意到随机变量$X$对每个博雷尔集$B$的原像$X^{-1}(B)$都属于事件空间, 也就是落在$\P$的定义域$\calF$中! 这样, 我们就可以测出$X^{-1}(B)$的概率:
$$ P(X^{-1}(B)) = P(\{\omg \in \Omg: X(\omg) \in B\}). $$
在推进之前, 我们先看几个例子, 熟悉一下符号. 
\begin{example}
    TBD
\end{example}
\begin{example}
    TBD
\end{example}

真正有趣的地方来了: $P(X^{-1}(B))$的形式让人联想到复合函数$f(g(x)) = f \circ g(x)$, 那我们可以定义$P \circ X^{-1}$吗? 画个交换图先: 
\begin{center}
    \begin{tikzcd}
    \calB_\R \arrow[rd, "\P \circ X^{-1}", labels=below left] \arrow[r, "X^{-1}"] 
    & \calF \arrow[d, "\P"] \\
    & \R
\end{tikzcd}
\end{center}
现在解释一下$X^{-1}: \calB_\R \to \calF$. 我们知道, $X$是映射并不意味着$X$存在逆映射, 但是$X$的原像总是存在的, 记号为$X^{-1}(B)$, 其中$B \in \calB_\R$. 又因为$X^{-1}(B) \in \calF$, 我们可以将$X^{-1}$看作集族$\calB_\R$到$\calF$的\textbf{映射}. 容易验证, $\P \circ X^{-1}$是从$\calB_\R$到$\R$的映射, 我们记为$$\mu = \P \circ X^{-1}.$$
更过分的事情来了: $\mu$是$\calB_\R$上的\textit{概率测度}!
\begin{exercise}
    证明$\mu$是一个概率测度.
\end{exercise}
由于$\mu$是一个有限博雷尔测度, $\mu$对应了唯一的分布函数. 概率学家和分析学家在此达成了少有的共识: 实分析中的分布函数在概率论中也叫分布函数. 
我们定义随机变量$X$的\textbf{分布函数}(distribution function)为
$$ F(x) = \P(X^{-1}(-\infty, x]) := \P(X \leq x). $$
注意, $\P(X \leq x)$这种写法是概率论特有的记号, 看起来像变量$X$小于等于一个$x$(我第一次学概率论的时候还真是这么理解的), 但实际上是可测函数$X$的一个原像. 
\begin{exercise}
    证明分布函数$F$满足如下性质:
    \begin{enumerate}
    \item $F$单调递增.
    \item $\lim_{x \to \infty}F(x) = 1, \lim_{x \to -\infty}F(x) = 0$.
    \item $F$右连续.
    \item $\P(X = x) = F(x) - F(x^-)$. 
    \end{enumerate}
\end{exercise}

\begin{example}
    若函数$F$满足
     \begin{enumerate}
    \item $F$单调递增,
    \item $\lim_{x \to \infty}F(x) = 1, \lim_{x \to -\infty}F(x) = 0$,
    \item $F$右连续,
    \end{enumerate}
    则$F$是某个随机变量的分布函数. 
\end{example}