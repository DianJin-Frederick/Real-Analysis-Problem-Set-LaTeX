\section{$\R$上的博雷尔测度与分布函数}
\begin{exercise}
    设$F$单调递增且右连续, 令$\mu_F$为分布函数$F$对应的测度. 证明
    \begin{enumerate}
    \item $\mu_F(\{a\})=F(a)-F(a^-)$,
    \item $\mu_F([a,b])=F(b)-F(a^-)$,
    \item $\mu_F([a,b))=F(b^-) - F(a^-)$,
    \item $\mu_F((a,b))=F(b^-)-F(a)$. 
    \end{enumerate}
\end{exercise}
\begin{proof}
    $\mu_F$与$F$以左开右闭区间为纽带: $\mu_F((a,b]) = F(b)-F(a)$, 所以我们要想法设法将题中的集合转化到左开右闭区间上. 
    \begin{enumerate}
    \item 因为$\{a\} = \bCap{n=1}{\infty}(a-1/n, a]$, 所以由测度的单调收敛定理得
    $$\mu_F\br{\bCap{n=1}{\infty}(a-1/n, a]} = \lim_{n \to \infty}\mu_F((a-1/n,a])
    = \lim_{n \to \infty}[F(a)-F(a-1/n)] = F(a) - F(a^-). $$
    \item $[a,b] = \{a\} \cup (a,b]$;
    \item $[a,b) = [a,b] \setminus \{b\}$;
    \item $(a,b) = (a,b] \setminus \{b\}$.
    \end{enumerate} \qed 
\end{proof}
由该练习立刻可得:
\begin{exercise}
    $F$连续当且仅当$\mu_F(\{a\}) = 0$对所有单点集$\{a\} \subset \R$都成立. 
\end{exercise}

\begin{example}\label{Cantor_measure}\footnote{部分参考UW-Madison博士资格考试题}
    设$\sigma$是$[0,1]$上的博雷尔测度, 且满足以下条件:
    \begin{enumerate}
    \item $\sigma([0,1]) = 1$;
    \item $\sigma([1/3, 2/3]) = 0$;
    \item $\sigma([a,b]) = \sigma([1-b,1-a])~\forall 0 \leq a < b \leq 1$;
    \item $\sigma([3a, 3b]) = 2\sigma([a,b])$对所有满足$0 \leq 3a < 3b \leq 1$的$a,b$都成立. 
    \end{enumerate}
    称这个$\sigma$为$[0,1]$上的$\frac{1}{3}$-康托测度. 
    \begin{enumerate}
        \item 求$\sigma([0, 1/8])$.
        \item 证明康托-勒贝格函数$F$对应的博雷尔测度$\mu_F$满足$\sigma$的4条性质(将题中4条性质中的$\sigma$替换成$\mu_F$)
        \item 证明$\sigma$对应的分布函数$F$就是康托-勒贝格函数
    \end{enumerate}
\end{example}
%\begin{remark}
    %本题第二问较难, 如果你对抽象测度的构造过程不熟悉, 可以将第二问改成``", 这样就可以避开唯一性的讨论.  
%\end{remark}
\begin{proof}
    \textbf{(1)}
    由$\sigma([0, 1/3]) = \sigma([2/3, 1])$以及
    $$1=\sigma([0,1]) = \sigma([0, 1/3]) + \sigma([2/3, 1]) + \sigma([1/3, 2/3])$$
    得$$\sigma([0, 1/3]) = \sigma([2/3, 1]) = \frac{1}{2}. $$
    由第4条性质得
    $$ \sigma\br{\left[ \frac{1}{9},\frac{2}{9} \right]} = 
       \frac{1}{2}\sigma\br{\left[ 3 \times \frac{1}{9},3 \times \frac{2}{9} \right]} = \frac{1}{2}\sigma\br{\left[ \frac{1}{3},\frac{2}{3} \right]} = 0, $$
    且$$ \sigma\br{\left[ 0,\frac{1}{9} \right]} = \frac{1}{2} \sigma\br{\left[ 0, 3 \times \frac{1}{9} \right]} = \frac{1}{2} \times \frac{1}{2} = \frac{1}{4}. $$
    所以
    $$\sigma\br{\left[ 0,\frac{1}{8} \right]} = \sigma\br{\left[ 0,\frac{1}{9} \right]} + \sigma\br{\left[ \frac{1}{9},\frac{1}{8} \right]} = \frac{1}{4} + 0 = \frac{1}{4}. $$
    \textbf{(2)} 接下来, 我们检验: 
    \begin{itemize}
    \item $F(1) - F(0) = 1$;
    \item $F(2/3) - F(1/3) = 0$;
    \item $F(b) - F(a) = F(1-a) - F(1-b)$;
    \item $F(3b) - F(3a) = 2(F(b) - F(a))$.
    \end{itemize}
    前两条不言自明. 先将$F$限制在康托集$\calC$上. 
    {\everymath{\displaystyle}
    记$a = \Sum{k=1}{\infty}\frac{a_k}{3^k}, b = \Sum{k=1}{\infty}\frac{b_k}{3^k}$, 
    其中$a_k, b_k \in \{0, 2\}$, 则
    $$ 1-a = \Sum{k=1}{\infty}\frac{2}{3^k} - \Sum{k=1}{\infty}\frac{a_k}{3^k}
    = \Sum{k=1}{\infty}\frac{2-a_k}{3^k}, \quad 
    1-b = \Sum{k=1}{\infty}\frac{2}{3^k} - \Sum{k=1}{\infty}\frac{b_k}{3^k}
    = \Sum{k=1}{\infty}\frac{2-b_k}{3^k}.
    $$
    于是
    \begin{align*}
    &F(1-a) - F(1-b) = \Sum{k=1}{\infty}\frac{1-a_k/2}{3^k} - \Sum{k=1}{\infty}\frac{1-b_k/2}{3^k} = \Sum{k=1}{\infty}\frac{(a_k-b_k)/2}{3^k} \\
    &F(b)-F(a) = \Sum{k=1}{\infty}\frac{b_k/2}{3^k}-\Sum{k=1}{\infty}\frac{a_k/2}{3^k} = \Sum{k=1}{\infty}\frac{(a_k-b_k)/2}{3^k}.
    \end{align*}
    同理可证$F(3b) - F(3a) = 2(F(b) - F(a))$.
    若$a,b \in [0,1] \setminus \calC$, 则由先前的练习, 有
    $$F(1-a) =  \sup_{s \geq a, s \in \calC} F(1-s) = \inf_{s \leq a, s \in \calC} F(1-s).$$ 
    这样做是为了将$F(a)$和$F(1-a)$中上下确界的指标集统一起来. 
    接下来我们利用确界的定义证明两个方向的不等式. 
    
    设$\eps > 0$, 则由$F(a), F(b)$的上确界定义知存在$x_0, y_0 \in \calC, x_0 \leq a, y_0 \leq b$, 使得
    $$F(a) \geq F(x_0) > F(a)-\eps, \quad F(b) \geq F(y_0) > F(b)-\eps. $$
    这里的$x_0, y_0$只是一个基准. 更进一步, 由$F$的单调性得
    $$F(a) \geq F(x) > F(a)-\eps, \quad F(b) \geq F(y) > F(b)-\eps \quad \text{对所有 }x \in [x_0,a] \cap \calC, y \in [y_0, b] \cap \calC \text{ 都成立}. $$
    
    于是$$ -F(a) \leq -F(x) < \eps - F(a), $$
    从而
    $$F(1-x)-F(1-y) = F(y)-F(x) > F(b)-F(a)-\eps. $$
    移项, 得
    $$F(1-x) > F(b)-F(a) + F(1-y) - \eps, $$
    
    对$y$取下确界, 根据$F(1-b) = \inf_{y \leq b, y \in \calC}F(1-y)$得
    $$F(1-x) \geq F(b)-F(a) + F(1-b) - \eps, $$ 再对$x$取下确界, 得
    $$F(1-a) - F(1-b) \geq F(b)-F(a)-\eps. $$ 因为$\eps$是任意的, 所以
    $$F(1-a) - F(1-b) \geq F(b) - F(a). $$
    反过来, 对任意的$\eps > 0$, 存在$x_0, y_0 \in \calC, x_0 \leq a, y_0 \leq b$, 使得
    $$F(1-a) \geq F(1-x_0) > F(1-a) - \eps, \quad F(1-b) \geq F(1-y_0) > F(1-b) $$
    由于$F(1-t)$是关于$t$的单调递减函数, 则对所有$x \in [x_0,a] \cap \calC, y \in [y_0, b] \cap \calC$都有
    $$F(1-a) \geq F(1-x) > F(1-a)-\eps, \quad F(1-b) \geq F(1-y) > F(b)-\eps. $$
    类似地, 我们有
    $$F(1-a)-F(1-b) \leq F(b) - F(a). $$
    同理可证, 若$a,b \in [0,1] \setminus \calC$, 则$F(3b)-F(3a) = 2(F(b)-F(a))$.
    所以$F$对应的博雷尔测度$\mu_F$满足$\sigma$的4个条件.} \\
    \textbf{(3)}   
    记$F$为康托-勒贝格函数, 我们已知$F$是$[0,1]$上单调增的连续函数, 所以有一个唯一的博雷尔测度$\mu_F$与之对应.
    你可能会想: 满足题中4个条件的博雷尔测度$\sigma$是唯一的吗? 
    因为$[0,1]$上的博雷尔$\sigma$-代数可由$\calE_1 = \{[a,b]: 0 \leq a < b \leq 1 \}$生成, 所以$\sigma$完全由$\sigma([a,b])$决定(详见测度的构造过程),
    %而$\sigma$在区间上的值由完全由其在康托集构造第$k$步中$C_k$的连通分支上的值所决定. 例如:
    %\begin{itemize}
        %\item 计算$\sigma([0,1/2])$, 只需用到第一步的$\sigma([0,1/3])=1/2$;
        %\item 计算$\sigma([0,1/8])$只需用到$C_2$.
    %\end{itemize}
    回顾康托集的构造过程, 我们在第$k$步得到了$2^k$个长度为$3^{-k}$的区间的无交并, 记为$C_k$. 记这些单独的闭区间(即连通分支)为$C_{k,j}, 1 \leq j \leq 2^k$. 根据$\sigma$的后两条性质以及数学归纳法, 我们有
    \begin{align*}
        &\sigma(C_1^c) = \sigma((1/3, 2/3)) = 0, \\
        &\sigma(C_2^c) = \sigma((1/9, 2/9) \cup (1/3, 2/3) \cup (7/9,8/9)) = 0, \\
        &\cdots \\
        &\sigma(C_k^c) = 0, \\
        &\cdots
    \end{align*}
    于是, $\sigma\br{\bCup{k=1}{\infty}C_k^c} = \sigma\br{ (\bCap{k=1}{\infty}C_k)^c } = \sigma([0,1] \setminus \calC) = 0$, 且对每个固定的$k$, 有
    $$\sigma(C_{k,j}) = \frac{1}{2^k} \quad (1 \leq j \leq 2^k). $$
    这样, $\sigma$在集族$\calE_2 = \{C_{k,j}: k \in \N, 1 \leq j \leq 2^k \}$上是唯一确定的.
    接下来, 我们只需说明: 如果$\sigma$在$\calE_2$上唯一确定, 则$\sigma$在$\calE_1$上唯一确定.
    我们先证明单点集的$\sigma$-测度为$0$. 
    \begin{itemize}
        \item 如果$a \in \calC$, 则$a \in C_k~\forall k$, 所以
        $$\sigma(\{a\}) \leq \sigma(C_k) = \br{\frac{2}{3}}^k ~\forall k, $$
        令$k \to \infty$得$\sigma(\{a\}) = 0$.
        \item 如果$a \notin \calC$, 则$$\sigma(\{a\}) \leq \sigma([0,1] \setminus \calC) = 0. $$ 
    \end{itemize}
    
    
    对任意的闭区间$[a,b]$, 我们可以将其拆分成两部分:
    $[a,b] = ([a,b] \cap \calC) \sqcup ([a,b] \cap \calC^c)$, 故
    \begin{align*}
        \sigma([a,b])
        &= \sigma([a,b] \cap \calC) + \sigma([a,b] \cap \calC^c) \\
        &= \sigma([a,b] \cap \calC) \\
        &= \sigma\br{\bCap{k=1}{\infty}[a,b] \cap C_k } \\
        &= \lim_{k \to \infty} \sigma([a,b] \cap C_k)
    \end{align*}
    现在, 我们来研究如何计算$\sigma([a,b] \cap C_k)$. 
    通过平移, $\sigma([a,b]) = \sigma([0, b-a])$, 
    所以我们不妨研究闭区间$[0,a], a > 0$.
    现在, 对$a$进行讨论:
    \begin{enumerate}
        \item 若$a \in (1/3, 2/3)$, 则$[0,a] \supset C_{1,1} = [0, 1/3]$, 所以$\sigma([0,a])=\sigma(C_{1,1}) = 1/2$. 
        \item 若$a \in [0, 1/3]$, 则$a \in [0, 1/9] \cup (1/9, 2/9) \cup [2/9, 1/3]$.
        \item 若$a \in [2/3, 1]$, 则只需考虑$[0, a] \cap [2/3, 1]$, 这跟$a \in [0, 1/3]$的情形没有区别. 所以接下来我们都考虑$a \in [0, 2/3^n]$的情形. 
        \item 由于$a>0$, 所以必定存在$k \in \N$, 使得$a \in (1/3^k, 2/3^k)$, 故$[0,a] \supset [0, 1/3^k] = C_{k, 1}$, 则$\sigma([0,a]) = \sigma(C_{k,1})$. 
        
    \end{enumerate}
    所以$\sigma$在闭区间上的值由$\sigma$在$\calE_2$上的值唯一确定. 再详细一点: 若$\mu$是另一个满足题中4个条件的博雷尔测度, 则由完全类似的过程可得$\mu([0,a]) = \mu(C_{k,j})$, 而那4条性质又告诉我们, $\mu(C_{k,j}) = 3^{-k}$, 所以
    $$\mu(C_{k,j}) = \sigma(C_{k.j}), $$ 从而$$\mu([0,a]) = \sigma([0,a]), $$
    故$$\mu([a,b]) = \sigma([a,b]). $$
    
    %这样, $\sigma$对应的唯一的分布函数$F_\sigma$就是连续的, 且必满足
    %\begin{align*}
        %&\sigma([a,b])=F_\sigma(b) - F_\sigma(a), \\
        %&F_\sigma(b) - F_\sigma(a) = F_\sigma(1-a) - F_\sigma(1-b) \\
        %&F_\sigma(3b) - F_\sigma(3a) = 2(F_\sigma(b) - F_\sigma(a)).
    %\end{align*}
    %这4条性质足以让我们对每个$k \in \N$, 计算出$\sigma$在$C_k$的连通分支上的值. 因而我们只需要证明
    现在, 我们证明康托-勒贝格函数$F$对应的博雷尔测度$\mu_F$满足$\sigma$的4个条件. 由于$F$连续, 所以
    $$F(b)-F(a) = \mu_F([a,b]). $$ 
\end{proof}

\begin{exercise}
    书接上例, 在不验证康托-勒贝格函数$F$对应的博雷尔测度满足那4条性质的情况下, 证明$\sigma$对应的分布函数$F_\sigma$是康托-勒贝格函数.
\end{exercise}
\begin{proof}
    由博雷尔测度的分布函数的定义知, $F_\sigma(a) = \sigma((0, a])$.
    由上例的证明过程知$\sigma$在单点集处的测度为$0$, 所以$F_\sigma(a) = \sigma([0,a])$.
    一般地, 我们有$F_\sigma(b)-F_\sigma(a) = \sigma([a,b])$. 接下来分$a \in \calC$和$a \notin \calC$两种情况讨论. 若$a \in \calC$, 则利用三进制小数证明$F_\sigma(a) = F(a)$. 细节留作练习. \qed
\end{proof}
