\section{从博雷尔测度到勒贝格测度}
我们将Carath\'eodory条件与先前做过的例题 \ref{Caratheodory_thm} 结合起来, 实际上就得到了测度论中的Carath\'eodory定理:
\begin{theorem}[Carath\'eodory]
    设$X$是一个集合, $\mu^*$是$X$上的一个外测度, 记$\calM^*$为所有满足Carath\'eodory条件的集合构成的集族, 则$\calM^*$是一个完备的$\sigma$-代数. 
\end{theorem}
\begin{exercise}
    证明Carath\'eodory定理中的完备性. 
\end{exercise}
以此定理为基础, 我们得以笼统地叙述构造一个测度的全过程\footnote{更紧凑的表述详见Real Analysis, Folland, Theorem 1.14 以及实分析正课}.:
     设$X$是一个集合. 我们从$X$上的一个初等类$\calE$以及其上的体积函数$\rho$出发, 将$\rho$延拓至$\calE$生成的代数$\calA$. 我们知道, $\calA$由$\calE$中的集合的有限无交并构成, 即
     $$\calA = \cBr{\bigsqcup_{n=1}^{N}E_n: E_n \in \calE, N \in \N}, $$
     所以对$A \in \calA$, 有$A = \bigsqcup_{n=1}^{N}E_n$, 其中$E_n \in \calE$. 故可令
     $$\rho(E) = \Sum{n=1}{N}\rho(E_n). $$ 接下来, 我们在$\calA$上定义一个预测度. 设$A \in \calA$, 令$\mu_0(A) = \rho(A)$. 
     若$A_1, A_2, \cdots \in \calA$互不相交且$\bCup{n=1}{\infty} \in \calA$, 则令
     $$ \mu_0\br{\bCup{n=1}{\infty}A} = \Sum{n=1}{\infty}\rho(A_n). $$
     这样, $\mu_0$便成为一个预测度, 它可以引导出外测度$\mu^*$, 即
    $$\mu^*(E) = \inf \curlBrace{\Sum{n=1}{\infty}\mu_0(A_j): E \subset \bCup{n=1}{\infty}A_j, A_j \in \calA}. $$
    注意到$\calA$中的每个集合都满足Carath\'eodory条件, 因此$\calA \subset \calM^*$, 故$\sigma(\calA) \subset \calM^*$. 那么, $\mu^*$限制在$\sigma(\calA)$(由$\calA$生成的$\sigma$-代数)上就是一个测度. 
    
\begin{remark}
    由于初等类的结构过于简单, 在实际应用中不费吹灰之力便可跨入代数. 例如$\R$上的初等类$\{(a,b]: a<b\}$, 轻而易举地就能延拓为代数, 并且在其上定义预测度
    $$ \mu_0\br{\bigsqcup_{n=1}^{\infty}(a_n, b_n]} = \Sum{n=1}{\infty}|(a_n, b_n]|. $$
\end{remark}

我们很快就能发现问题: Carath\'eodory定理说$\mu^*$限制在$\calM^*$上是一个(完备)测度, 而测度的构造定理说$\mu^*$限制住$\sigma(\calA)$上是一个测度. 一个是从$\calP(x)$到$\calM^*$(从上往下), 一个是从$\calA$到$\sigma(A)$(从下往上). 那$\sigma(\calA)$和$\calM^*$之间仅仅拥有简单的包含关系吗?

\subsection{完备化的两条路径}
回顾正课的知识, 我们知道$\sigma(\calA)$就是$\R$上的博雷尔$\sigma$-代数$\calB$, $\calM^*$就是$\R$上的勒贝格$\sigma$-代数$\calL$, 而且$\calL$是$\calB$的完备化. 
这由正则性质可以立刻得到. 
\subsubsection*{从拓扑的角度切入: 正则性}
设$E \subset \R$是一个勒贝格可测集, 则$E$可以写成如下形式:
$$E = F \cup N, $$
其中$F$为$F_\sigma$集, $N$的勒贝格测度为$0$. 因为$F_\sigma$集是闭集的可数并, 所以$F \in \calB$. 

接下来, 我们说明$N$一定是某个博雷尔零测集$M$的子集. 因为$N$勒贝格可测, 所以对每个$n \in \N$, 存在开集$G_n \supset N$, 使得$$m(G_n) < \frac{1}{n}. $$
令$G = \bCap{n=1}{\infty}G_n$, 则$G \supset N$, $G \in \calB$, 且$G$的博雷尔测度为$0$. 于是, 每个勒贝格可测集都可以写成一个博雷尔集和一个博雷尔零测集的子集的并, 所以$\calL$是$\calB$的完备化. 


但是, 对没有配备拓扑的一般集合, 我们就无法谈正则性质.
\subsubsection*{从测度论的角度切入}
我们从测度的构造过程入手, 探究$\sigma(\calA)$与$\calM^*$的关系. 

\begin{exercise}\footnote{Real Analysis, Folland, Exercise 1.22}\label{ex1-22}
    Let $\mathcal{A} \subset \mathcal{P}(X)$ be an algebra, $\mathcal{A}_\sigma$ the collection of countable unions of sets in $\mathcal{A}$, and $\mathcal{A}_{\sigma \delta}$ the collection of countable intersections of sets in $\mathcal{A}_\sigma$. Let $\mu_0$ be a premeasure on $\mathcal{A}$ and $\mu^*$ the induced outer measure.
    \begin{enumerate}
    \item For any $E \subset X$ and $\epsilon>0$ there exists $A \in \mathcal{A}_\sigma$ with $E \subset A$ and $\mu^*(A) \leq$ $\mu^*(E)+\epsilon$.
    \item If $\mu^*(E)<\infty$, then $E$ is $\mu^*$-measurable implies that there exists $B \in \mathcal{A}_{\sigma \delta}$ with $E \subset B$ and $\mu^*(B \backslash E)=0$.
    \item If $\mu_0$ is $\sigma$-finite, the restriction $\mu^*(E)<\infty$ in (b) is superfluous. 
    \end{enumerate}
\end{exercise}
\begin{proof}
    By the definition of an outer measure, there exists $Q_n \in \calA$ with $E \subset \bigunion{n=1}{\infty}Q_n$ such that 
    $$\mu^*\Brace{\bigunion{n=1}{\infty}Q_n} \leq \mu^*(E) + \eps, $$
    set $A = \bigunion{n=1}{\infty}Q_n$ completes part (1). 

    For each $n \in \N$ there exists an $A_n \in \calA_\sigma \subset \sigma(\calA)$ such that 
    $\mu^*(A_n) \leq \mu^*(E) + 1/n$,
    then 
    $$\mu^*\Brace{\bigintersect{n=1}{\infty}A_n}
    = \lim_{n \to \infty}\mu^*(A_n) \leq \mu^*(E).$$
    The reverse inequality is obvious. Set $B = \bigintersect{n=1}{\infty}A_n$ and thus $\mu^*(B \setminus E) = 0$. 
    If $\mu_0$ is $\sigma$-finite, then $X = \bigunion{n=1}{\infty}X_n$ with $\mu^*(X_n) < \infty$, so we can write 
    $E = \bigunion{n=1}{\infty}X_n \cap E$, where $E_n = X_n \cap E$. For each $E_n$ we have $B_n \supset E$ with $\mu^*(B_n \setminus E_n) = 0$, hence 
    $$\mu^*\Brace{\bigunion{n=1}{\infty}B_n \setminus \bigunion{n=1}{\infty}E_n} = 
    \mu^*\Brace{\bigunion{n=1}{\infty}(B_n \setminus E_n)} = 0.$$
\end{proof}

\begin{exercise}
    Let $(X, \mathcal{M}, \mu)$ be a measure space, $\mu^*$ the outer measure induced by $\mu$, $\mathcal{M}^*$ the $\sigma$-algebra of $\mu^*$-measurable sets, and $\bar{\mu}=\mu^* \mid \mathcal{M}^*$.
    If $\mu$ is $\sigma$-finite, then $\bar{\mu}$ is the completion of $\mu$.
\end{exercise}
\begin{proof}
    Let $E \in \calM^*$, then there exists $B \in \sigma(\calA)$ with $\mu^*(B \setminus E) = 0$, so $E = B \cup (B \setminus E)$
\end{proof}

\begin{definition}
    Let $F: \R \to \R$ be any increasing and right continuous function. We call $\cl{\mu}_F$ the Lebesgue-Stieltjes measure asscoiated to $F$, and usually denote this complete measure also by $\mu_F$. 
\end{definition}


\subsection{Junction: Carath\'eodory and Lebesgue}
In Real Analysis I, we derive Lebesgue measure by restricting the outer measure to a smaller family of sets by defining $E$ to be Lebesgue measurable if 
$$\text{for every }\eps>0 \text{ there exists an open set }\calO \supset E \text{ with }m^*(\calO \setminus E) < \eps.$$
Then we showed that under this condition, the family of Lebesgue measurable sets forms a $\sigma$-algebra. 

In Real Analysis III we use the Carath\'eodory-style approach to obtain the Lebesgue $\sigma$-algebra by declaring $E$ is Lebesgue measurable if 
$$m^*(A) = m^*(A \cap E) + m^*(A \cap E^c) \quad \forall A \subset \R, $$
which is elegant and easy to manipulate. 

Now we show that the above two conditions are equivalent. 
First we assume $E$ satisfies the Carath\'eodory condition ($E \in \calM_\mu$) and derive the first regularity property. 
We begin by a lemma modifying h-intervals to open intervals. 
\begin{lemma}
    Let $\mu$ be a fixed Lebesgue-Stieltjes measure with domain $\calM_\mu$. 
    For any $E \in \calM_\mu$, 
    $$\mu(E) = \inf \curlBrace{\Sum{j=1}{\infty}
      \mu((a_j, b_j)): E \subset \bigunion{j=1}{\infty}(a_j,b_j)}. $$
\end{lemma}
\begin{proof}
    Let us call the quantity on the right $\nu(E)$. Suppose $E \subset \bigcup_1^{\infty}(a_j, b_j)$. Each $\left(a_j, b_j\right)$ is a countable disjoint union of h-intervals $I_j^k(k=1,2, \ldots)$; specifically, $I_j^k=\(c_j^k, c_j^{k+1}]$ where $\left\{c_j\right\}$ is any sequence such that $c_j^1=a_j$ and $c_j^k$ increases to $b_j$ as $k \rightarrow \infty$. Thus $E \subset \bigcup_{j, k=1}^{\infty} I_j^k$, so
    $$
    \sum_1^{\infty} \mu\left(\left(a_j, b_j\right)\right)=\sum_{j, k=1}^{\infty} \mu(I_j^k) \geq \mu(E),
    $$
    and hence $\nu(E) \geq \mu(E)$. On the other hand, given $\epsilon>0$ there exists $\left\{\left(a_j, b_j\right]\right\}_1^{\infty}$ with $E \subset \bigcup_1^{\infty}\left(a_j, b_j\right]$ and $\sum_1^{\infty} \mu\left(\left(a_j, b_j\right]\right) \leq \mu(E)+\epsilon$, and for each $j$ there exists $\delta_j>0$ such that $F\left(b_j+\delta_j\right)-F\left(b_j\right)<\epsilon 2^{-j}$. Then $E \subset \bigcup_1^{\infty}\left(a_j, b_j+\delta_j\right)$ and
    $$
    \sum_1^{\infty} \mu\left(\left(a_j, b_j+\delta_j\right)\right) \leq \sum_1^{\infty} \mu\left(\left(a_j, b_j\right]\right)+\epsilon \leq \mu(E)+2 \epsilon,
    $$
    so that $\nu(E) \leq \mu(E)$. \qed
\end{proof}
\begin{theorem}
    Let $\mu$ be a fixed Lebesgue-Stieltjes measure with domain $\calM_\mu$. 
    If $E \in \calM_\mu$, then 
    $$\mu(E) = \inf \{\mu(U): U \supset E, U \text{ is open}\}.$$
\end{theorem}
\begin{proof}
    For any $\epsilon>0$ there exist intervals $\left(a_j, b_j\right)$ such that $E \subset \bigcup_1^{\infty}\left(a_j, b_j\right)$ and $\mu(E) \leq \sum_1^{\infty} \mu\left(\left(a_j, b_j\right)\right)+\epsilon$. If $U=\bigcup_1^{\infty}\left(a_j, b_j\right)$ then $U$ is open, $U \supset E$, and $\mu(U) \leq \mu(E)+\epsilon$. On the other hand, $\mu(U) \geq \mu(E)$ whenever $U \supset E$, so the first equality is valid.  \qed
\end{proof}
The Lebesgue measure is a special case of Lebesgue-Stieltjes measure with $F(x) = x$, so we can apply the above results. 
Conversely, suppose that $E \subset X$ and $\text{for every }\eps>0 \text{ there exists an open set }U \supset E \text{ with }m^*(U \setminus E) < \eps.$ Then by a limiting argument we can find a $G_\d$ set $G$ with $E = G \setminus N$ and $m(N)=0$, thus $E = G \cup N \in \calM_\mu$. The proof is complete. 

\subsection{历史回眸: Jordan容度}
Jordan外容度和Lebesgue外测度的区别在与有限覆盖与可数覆盖. 
\begin{exercise}\footnote{Real Analysis, Stein, Exercise 1.14}
    设$E \subset \R$, 定义$E$的Jordan外容度为
    $$J^*(E) = \inf \curlBrace{\Sum{j=1}{N}|I_j|: E \subset \bCup{n=1}{N}I_j, N \in \N, I_j \text{ 为开区间}}.$$
    \begin{enumerate}
    \item 证明$J^*(E) = J^*(\cl{E})$. 
    \item 找出一个可数集$E \subset [0,1]$, 使得$J^*(E) = 1, m^*(E) = 0$. 
    \end{enumerate}
\end{exercise}