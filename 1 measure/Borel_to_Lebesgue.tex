\section{从博雷尔测度到勒贝格测度}
我们将Carath\'eodory条件与先前做过的例题 \ref{Caratheodory_thm} 结合起来, 实际上就得到了测度论中的Carath\'eodory定理:
\begin{theorem}[Carath\'eodory]
    设$X$是一个集合, $\mu^*$是$X$上的一个外测度, 记$\calM^*$为所有满足Carath\'eodory条件的集合构成的集族, 则$\calM^*$是一个完备的$\sigma$-代数. 
\end{theorem}
\begin{exercise}
    证明Carath\'eodory定理中的完备性. 
\end{exercise}
以此定理为基础, 我们得以笼统地叙述构造一个测度的全过程\footnote{更紧凑的表述详见Real Analysis, Folland, Theorem 1.14 以及实分析正课}.:
     设$X$是一个集合. 我们从$X$上的一个初等类$\calE$以及其上的体积函数$\rho$出发, 将$\rho$延拓至$\calE$生成的代数$\calA$. 我们知道, $\calA$由$\calE$中的集合的有限无交并构成, 即
     $$\calA = \cBr{\bigsqcup_{n=1}^{N}E_n: E_n \in \calE, N \in \N}, $$
     所以对$A \in \calA$, 有$A = \bigsqcup_{n=1}^{N}E_n$, 其中$E_n \in \calE$. 故可令
     $$\rho(E) = \Sum{n=1}{N}\rho(E_n). $$ 接下来, 我们在$\calA$上定义一个预测度. 设$A \in \calA$, 令$\mu_0(A) = \rho(A)$. 
     若$A_1, A_2, \cdots \in \calA$互不相交且$\bCup{n=1}{\infty} \in \calA$, 则令
     $$ \mu_0\br{\bCup{n=1}{\infty}A} = \Sum{n=1}{\infty}\rho(A_n). $$
     这样, $\mu_0$便成为一个预测度, 它可以引导出外测度$\mu^*$, 即
    $$\mu^*(E) = \inf \curlBrace{\Sum{n=1}{\infty}\mu_0(A_j): E \subset \bCup{n=1}{\infty}A_j, A_j \in \calA}. $$
    注意到$\calA$中的每个集合都满足Carath\'eodory条件, 因此$\calA \subset \calM^*$, 故$\sigma(\calA) \subset \calM^*$. 那么, $\mu^*$限制在$\sigma(\calA)$(由$\calA$生成的$\sigma$-代数)上就是一个测度. 
    
\begin{remark}
    由于初等类的结构过于简单, 在实际应用中不费吹灰之力便可跨入代数. 例如$\R$上的初等类$\{(a,b]: a<b\}$, 轻而易举地就能延拓为代数, 并且在其上定义预测度
    $$ \mu_0\br{\bigsqcup_{n=1}^{\infty}(a_n, b_n]} = \Sum{n=1}{\infty}|(a_n, b_n]|. $$
\end{remark}

我们很快就能发现问题: Carath\'eodory定理说$\mu^*$限制在$\calM^*$上是一个(完备)测度, 而测度的构造定理说$\mu^*$限制住$\sigma(\calA)$上是一个测度. 一个是从$\calP(x)$到$\calM^*$(从上往下), 一个是从$\calA$到$\sigma(A)$(从下往上). 那$\sigma(\calA)$和$\calM^*$之间仅仅拥有简单的包含关系吗?

\subsection{完备化的两条路径}
回顾正课的知识, 我们知道$\sigma(\calA)$就是$\R$上的博雷尔$\sigma$-代数$\calB$, $\calM^*$就是$\R$上的勒贝格$\sigma$-代数$\calL$, 而且$\calL$是$\calB$的完备化. 
这由正则性质可以立刻得到. 
\subsubsection*{从拓扑的角度切入: 正则性}
设$E \subset \R$是一个勒贝格可测集, 则$E$可以写成如下形式:
$$E = F \cup N, $$
其中$F$为$F_\sigma$集, $N$的勒贝格测度为$0$. 因为$F_\sigma$集是闭集的可数并, 所以$F \in \calB$. 

接下来, 我们说明$N$一定是某个博雷尔零测集$M$的子集. 因为$N$勒贝格可测, 所以对每个$n \in \N$, 存在开集$G_n \supset N$, 使得$$m(G_n) < \frac{1}{n}. $$
令$G = \bCap{n=1}{\infty}G_n$, 则$G \supset N$, $G \in \calB$, 且$G$的博雷尔测度为$0$. 于是, 每个勒贝格可测集都可以写成一个博雷尔集和一个博雷尔零测集的子集的并, 所以$\calL$是$\calB$的完备化. 


但是, 对没有配备拓扑的一般集合, 我们就无法谈正则性质.
\subsubsection*{从测度论的角度切入}
我们从测度的构造过程入手, 探究$\sigma(\calA)$与$\calM^*$的关系. 

\begin{exercise}\footnote{Real Analysis, Folland, Exercise 1.22}\label{ex1-22}
    设$\calA$是$X$上的一个代数, $\calA_\sigma$为$\calA$中集合的所有可数并构成的集族, 即
    $$\calA_\sigma = \left\{\bCup{n=1}{\infty}A_n: A_n \in \calA~\forall n \in \N \right\}.$$
    设$A_{\sigma \d}$是由$A_\sigma$中集合的所有可数交构成的集族, 即
    $$\calA_\sigma = \left\{\bCap{n=1}{\infty}B_n: B_n \in \calA_\sigma~\forall n \in \N \right\}.$$ 设$\mu_0$是$\calA$上的一个预测度, $\mu^*$是其引导的外测度. 证明:
    \begin{enumerate}
        \item 对任意$E \subset X$与$\eps > 0$, 都存在$A \in \calA_\sigma$满足$E \subset A$和$\mu^*(A) \leq \mu^*(E) + \eps$. 
        \item 如$\mu^*(E) < \infty$且$E$是$\mu^*$-可测的, 则存在$B \in \calA_{\sigma \delta}$满足$E \subset B$且$\mu^*(B \setminus E) = 0$. 
    \end{enumerate}
\end{exercise}
\begin{proof}
    根据外测度的定义, 存在$Q_n \in \calA$满足$E \subset \bigunion{n=1}{\infty}Q_n$, 使得
    $$\mu^*\Brace{\bigunion{n=1}{\infty}Q_n} \leq \mu^*(E) + \eps. $$
    令$A = \bigunion{n=1}{\infty}Q_n$即可证明(1). 

    对每个$n \in \N$, 都存在$A_n \in \calA_\sigma \subset \sigma(\calA)$, 使得
    $\mu^*(A_n) \leq \mu^*(E) + 1/n$,
    则 
    $$\mu^*\Brace{\bigintersect{n=1}{\infty}A_n}
    = \lim_{n \to \infty}\mu^*(A_n) \leq \mu^*(E).$$
    反方向的不等式是显然的. 令$B = \bigintersect{n=1}{\infty}A_n$, 则有$\mu^*(B \setminus E) = 0$.
    因为$\mu_0$是$\sigma$-有限的, 我们可以把$X$写成$X = \bigunion{n=1}{\infty}X_n$, 其中$\mu^*(X_n) < \infty$, 
    故
    $E = \bigunion{n=1}{\infty}X_n \cap E$, 
    其中$E_n = X_n \cap E$. 对每个$E_n$, 我们有$B_n \supset E$, 使得$\mu^*(B_n \setminus E_n) = 0$, 于是
    $$\mu^*\Brace{\bigunion{n=1}{\infty}B_n \setminus \bigunion{n=1}{\infty}E_n} = 
    \mu^*\Brace{\bigunion{n=1}{\infty}(B_n \setminus E_n)} = 0.$$ \qed 
\end{proof}
\begin{exercise}
    设$(X, \mathcal{M}, \mu)$是一个测度空间, $\mu^*$是$\mu$引导的外测度, 
    $\mathcal{M}^*$是由$\mu^*$可测集构成的$\sigma$-代数, 令$\bar{\mu}=\mu^* \mid \mathcal{M}^*$.
    若$\mu$是$\sigma$-有限的, 则$\bar{\mu}$是$\mu$的完备化. 
\end{exercise}
\begin{proof}
    设$E \in \calM^*$, 则存在$B \in \sigma(\calA)$使得$\mu^*(B \setminus E) = 0$, 所以$E = B \cup (B \setminus E)$. \qed 
\end{proof}

\subsection{历史回眸: Jordan容度}
Jordan外容度和Lebesgue外测度的区别在于有限覆盖与可数覆盖. 
\begin{exercise}\footnote{Real Analysis, Stein, Exercise 1.14}
    设$E \subset \R$, 定义$E$的Jordan外容度为
    $$J^*(E) = \inf \curlBrace{\Sum{n=1}{N}|I_n|: E \subset \bCup{n=1}{N}I_n, N \in \N, I_n \text{ 为开区间}}.$$
    \begin{enumerate}
    \item 证明$J^*(E) = J^*(\cl{E})$. 
    \item 找出一个可数集$E \subset [0,1]$, 使得$J^*(E) = 1, m^*(E) = 0$. 
    \end{enumerate}
\end{exercise}
\begin{proof}
    \begin{enumerate}
    \item 这是典型的an $\eps$ of room问题, 关键在于研究包含$E$的开区间族与$\cl{E}$的开区间族的长度之差. 首先观察到覆盖$\cl{E}$的开区间族一定能覆盖$E$, 所以$J^*(E) \geq J(E)$. 
    
    设$\bCup{n=1}{N}I_n = \bCup{n=1}{N}(a_n,b_n) \supset E$, 令
    $$ A_n = \br{a_n-\frac{\eps}{2N}, b_n+\frac{\eps}{2N}}, $$
    则$$ \Sum{n=1}{N}|A_n| \leq \Sum{n=1}{N}|I_n| + \eps, $$
    所以$$J^*(\cl{E}) \leq \Sum{n=1}{N}|A_n| \leq \Sum{n=1}{N}|I_n| + \eps. $$
    对所有覆盖$E$的开区间族$\{I_n: n=1, \cdots, N; N \in \N\}$取下确界, 得
    $$ J^*(\cl{E}) \leq J^*(E) + \eps, $$
    因为$\eps>0$是任意的, $J^*(\cl{E}) \leq J^*(E)$. 
    \item 令$E = \Q$. 设$\bCup{n=1}{N}I_n \supset \Q$, 则由有理数的稠密性得$\bCup{n=1}{N}I_n = [0,1]$, 所以$J^*(E) = 1$. 
    \end{enumerate} \qed 
\end{proof}
