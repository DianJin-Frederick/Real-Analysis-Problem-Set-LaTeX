\section{康托集}
康托集一开始由康托提出, 目的是打破人们对于集合性质的一些常规认知, 
后来康托集也成为几何测度论中的一个重要研究对象. 该集合的勒贝格测度为$0$, 所以Lebesgue测度并不能很好地展现出Cantor集的特点. 今后我们会学习Hausdorff测度, 并且证明Cantor集的Hausdorff维数为 $\log 2 / \log 3$.
\subsection{定义与性质}
康托集有两个等价定义, 我们在正课中均已见过:
\begin{enumerate}
%\everymath{\displaystyle}
    \item 从$[0,1]$出发, 第一次挖掉$[\frac{1}{3}, \frac{2}{3}]$, 
    剩下来的集合记作$C_1$,
    第二次在$C_1$上挖掉$[\frac{1}{9}, \frac{2}{9}] \cup [\frac{7}{9}, \frac{8}{9}]$, 剩下来的集合记作$C_2$, 这样无限进行下去得Cantor集$\calC = \bigintersect{k=1}{\infty}C_k$.
    \item $[0,1]$中的实数可以写成三进制形式: $x = \Sum{k=1}{\infty}a_k/3^k$, 其中每个$a_k \in \{0,1,2\}$. Cantor集也可定义为
    $$\calC = \curlBrace{x=\Sum{k=1}{\infty}\frac{a_k}{3^k}: a_k \in \{0,2\}}.$$
\end{enumerate}
\begin{remark}
    根据无穷级数得知识可知
    $$\frac{1}{3} = \frac{1}{3^2}+\frac{1}{3^3}+\frac{1}{3^4}+\cdots,$$
    而$1/3$显然等于它自己, 于是同一个数就会有两种不同的三进制表示, 所以我们约定这种情况总是取无穷级数的表示.
\end{remark}
我们再复习一下Cantor集的性质:
\begin{property}
    \begin{enumerate}
    \item $\calC$是紧集;
    \item $m(\calC)=0$;
    \item $\calC$具有连续基数;
    \item $\calC$无处稠密, 即$\cl{\calC}^\circ=\varnothing$;
    \item $\calC$完全不连通(totally disconnected), 即任取$x,y \in \calC$, 总能找到$z \in (x,y): z \notin \calC$. 
    \item $\calC$无孤立点, 即$\calC$是一个完全(perfect)集.
    \end{enumerate}
\end{property}
\subsection{康托-勒贝格函数}
我们可以把$\calC$中的三进制小数映射到二进制小数. 
Cantor-Lebesgue函数的定义为$F:[0,1] \to [0,1]$,
$$F\Brace{\Sum{k=1}{\infty}\frac{a_k}{3^k}} = \Sum{k=1}{\infty}\frac{b_k}{2^k}, \quad 
\text{其中~}b_k = \frac{a_k}{2}, a_k \in \{0,1,2\}.$$

\begin{exercise}
    证明$F$是满射, 即对每个$y \in [0,1]$都存在$x \in \calC$使$F(x)=y$.
\end{exercise}
从该练习可以得出$\calC$具有连续基数. 

\begin{exercise} % Stein ex 1-21
    证明存在一个连续函数将一个勒贝格可测集映射至一个不可测集. (提示: 考虑$[0,1]$的一个不可测子集, 以及它在康托-勒贝格函数下的原像)
\end{exercise}


\subsection{广义康托集}
康托集的构造过程很容易推广: 改一改挖掉的区间数, 改一改挖去的区间长度, 就能得到许多类康托集(Cantor-like sets)(或称康托型集, Cantor-type sets).
同样, 将康托集的一些性质归纳出来进行排列组合, 比如``完全不连通的紧集"等等. 我们也可以将这些集合归于类康托集. 为和正课讲义保持一致, 我们将上面这两种集合统称为``广义康托集". 接下来我们看看广义康托集的应用. 

\begin{example} % Stein ex 1-20
    举例: $A, B \subset \R$为闭集且$m(A)=m(B)=0$, 但$m(A+B)>0$.
\end{example}
\begin{proof}
    令$A = \calC, B = \calC / 2$. 
    \qed
\end{proof}


\begin{example}
    构造一个博雷尔集$A \subset [0,1]$满足
    $$ 0<m(A \cap I)<m(I) \quad \text{对所有区间}~I \subset [0,1]~\text{都成立}. $$
\end{example}






\begin{example}
    构造一个$\R$中的博雷尔集$A$满足
    $$ 0<m(A \cap I)<m(I) \quad \text{对所有区间}~I~\text{都成立}. $$
\end{example}
\begin{solution}
    % 先证明对所有有理区间成立
    ``构造"和``对所有"明示了这道题的难度非同寻常, 我们必须简化``对所有区间"这个条件. 回想有理数集在实数集中的稠密性, 我们可以尝试用有理区间代替所有区间. 如果我们能构造出$A$使结论对所有有理区间都成立, 那么该结论也应该对所有区间都成立. 现在我们来论证这个陈述:

    设$\Q = \{r_1, r_2, \cdots\}$, 假设我们已经构造出了博雷尔集$A$满足
    $$0 < m(A \cap (r_i,r_j)) < r_j - r_i \quad \text{对所有}~r_i<r_j~\text{都成立}.$$
    现任取一区间$I=(a,b)$(只需考虑开区间即可, 为什么?), 则可以找到有理数$a<r_i<r_j<b$. 接下来,
    \begin{align*}
    m(A \cap I)
    &= m(A \cap (a,r_i)) + m(A \cap (r_i,r_j)) + m(A \cap (r_j,b)) \\
    &< (r_i-a) + (r_j - r_i) + (b-r_j) \\
    &= b-a,
    \end{align*}
    且$m(A \cap I) \geq m(A \cap (r_i-r_j)) > 0$, 故结论对所有区间$I$都成立. 

    至此, 我们已将题目简化为了``构造一个$\R$中的博雷尔集$A$满足$0<m(A \cap I)<m(I)$对所有有理区间$I$都成立."
    
\end{solution}

% https://www.jstor.org/stable/pdf/2975692.pdf?refreqid=excelsior%3A200cfff310a747c5a4b7f4620fd7f019&ab_segments=&origin=&acceptTC=1